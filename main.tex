\documentclass[12pt, a4paper]{article}

% ── Packages ──────────────────────────────────────────────────
\usepackage[utf8]{inputenc}
\usepackage[T1]{fontenc}
\usepackage{lmodern}
\usepackage[margin=1in]{geometry}
\usepackage{amsmath, amssymb, amsthm}
\usepackage{graphicx}
\usepackage{hyperref}
\usepackage{xcolor}
\usepackage{enumitem}
\usepackage{booktabs}
\usepackage{caption}
\usepackage{subcaption}
\usepackage{fancyhdr}
\usepackage{lipsum}  % for demo text — remove in real documents

% ── Page style ────────────────────────────────────────────────
\pagestyle{fancy}
\fancyhf{}
\rhead{\thepage}
\lhead{\leftmark}

% ── Hyperlink colors ─────────────────────────────────────────
\hypersetup{
    colorlinks=true,
    linkcolor=blue!70!black,
    citecolor=green!50!black,
    urlcolor=blue!60!black,
}

% ── Theorem environments ─────────────────────────────────────
\newtheorem{theorem}{Theorem}[section]
\newtheorem{lemma}[theorem]{Lemma}
\newtheorem{definition}[theorem]{Definition}

% ── Title ─────────────────────────────────────────────────────
\title{\textbf{My First LaTeX Document}\\[0.3em]
       \large A Starter Template for VS Code + GitHub}
\author{Your Name}
\date{\today}

% ══════════════════════════════════════════════════════════════
\begin{document}
\maketitle
\tableofcontents
\newpage

% ── Section 1 ─────────────────────────────────────────────────
\section{Introduction}

Welcome to your local \LaTeX{} environment! This setup gives you
an \textbf{Overleaf-like experience} right inside VS Code:

\begin{itemize}
    \item \textbf{Auto-compile on save} — just press \texttt{Ctrl+S}
    \item \textbf{Side-by-side PDF preview} — click the magnifying glass icon
    \item \textbf{SyncTeX} — \texttt{Ctrl+Click} on the PDF to jump to source
    \item \textbf{Git integration} — commit and push from the sidebar
    \item \textbf{Spell \& grammar check} — powered by LTeX
\end{itemize}

% ── Section 2 ─────────────────────────────────────────────────
\section{Mathematics}

Here's the famous Euler's identity:
\begin{equation}
    e^{i\pi} + 1 = 0
\end{equation}

And the quadratic formula:
\begin{equation}
    x = \frac{-b \pm \sqrt{b^2 - 4ac}}{2a}
\end{equation}

\begin{theorem}[Pythagorean Theorem]
    For a right triangle with legs $a$, $b$ and hypotenuse $c$:
    \[ a^2 + b^2 = c^2 \]
\end{theorem}

% ── Section 3 ─────────────────────────────────────────────────
\section{Tables and Figures}

\begin{table}[h]
    \centering
    \caption{A sample table}
    \begin{tabular}{lcc}
        \toprule
        \textbf{Feature} & \textbf{Overleaf} & \textbf{VS Code + LaTeX Workshop} \\
        \midrule
        Auto-compile    & \checkmark & \checkmark \\
        PDF preview     & \checkmark & \checkmark \\
        Git integration & Limited    & Full       \\
        Offline work    & \texttimes & \checkmark \\
        Free            & Partial    & \checkmark \\
        \bottomrule
    \end{tabular}
\end{table}

% ── Section 4 ─────────────────────────────────────────────────
\section{Getting Started with GitHub}

\begin{enumerate}
    \item Create a repository on \href{https://github.com}{GitHub}
    \item Add the remote: \texttt{git remote add origin <URL>}
    \item Stage, commit, and push:
    \begin{verbatim}
    git add .
    git commit -m "Initial commit"
    git push -u origin main
    \end{verbatim}
\end{enumerate}

Or simply use the \textbf{Source Control} panel in VS Code's sidebar!

% ── Section 5 ─────────────────────────────────────────────────
\section{Useful Shortcuts}

\begin{description}
    \item[\texttt{Ctrl+S}] Save and auto-compile
    \item[\texttt{Ctrl+Alt+V}] Open PDF preview
    \item[\texttt{Ctrl+Click} on PDF] Jump to source (SyncTeX)
    \item[\texttt{Ctrl+Shift+G}] Open Git panel
    \item[\texttt{Ctrl+Space}] Trigger LaTeX IntelliSense
\end{description}

\lipsum[1] % Remove this line — it's just filler text for demo

\end{document}
